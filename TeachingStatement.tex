\documentclass[11pt]{amsart}
\usepackage{fancyhdr}
\usepackage[margin=1in]{geometry}

\usepackage[page]{totalcount}

\usepackage{amsthm}
\theoremstyle{plain}
\newtheorem{theorem}{Theorem}
\newtheorem{question}{Question}

\usepackage[pdfpagelabels]{hyperref}
\hypersetup{colorlinks=true,urlcolor=blue,linkcolor=black,citecolor=black}

% \usepackage{fontspec}
%   \defaultfontfeatures{Ligatures=TeX}
%   \setmainfont{Times New Roman}
% \usepackage{setspace}
% \doublespacing

\pagestyle{fancy} \headheight 15pt \footskip 20pt

\parskip=12pt

\rhead{\thepage/\totalpages}
\chead{}
\lhead{Clontz Teaching Statement}
\rfoot{\today}
\cfoot{}
\lfoot{}

\newcommand{\HRule}{\rule{\linewidth}{0.5mm}}


%%%%%%%%%%%%%%%
% Definitions %
%%%%%%%%%%%%%%%

% Strategy uparrow shortcuts
\newcommand{\win}{\uparrow}
\newcommand{\prewin}{\underset{\text{pre}}{\uparrow}}
\newcommand{\markwin}{\underset{\text{mark}}{\uparrow}}
\newcommand{\tactwin}{\underset{\text{tact}}{\uparrow}}
\newcommand{\kmarkwin}[1]{\underset{#1\text{-mark}}{\uparrow}}
\newcommand{\ktactwin}[1]{\underset{#1\text{-tact}}{\uparrow}}
\newcommand{\codewin}{\underset{\text{code}}{\uparrow}}
\newcommand{\limitwin}{\underset{\text{limit}}{\uparrow}}
\newcommand{\notwin}{\not\uparrow}
\newcommand{\notprewin}{\underset{\text{pre}}{\not\uparrow}}
\newcommand{\notmarkwin}{\underset{\text{mark}}{\not\uparrow}}
\newcommand{\nottactwin}{\underset{\text{tact}}{\not\uparrow}}
\newcommand{\notkmarkwin}[1]{\underset{#1\text{-mark}}{\not\uparrow}}
\newcommand{\notktactwin}[1]{\underset{#1\text{-tact}}{\not\uparrow}}
\newcommand{\notcodewin}{\underset{\text{code}}{\not\uparrow}}
\newcommand{\notlimitwin}{\underset{\text{limit}}{\not\uparrow}}

\newcommand{\oneptcomp}[1]{#1^*}
\newcommand{\oneptlind}[1]{#1^\dagger}
% \newcommand{\sharp}[1]{#1^{\#}}

% Games
\newcommand{\gruConGame}[2]{Gru_{O,P}(#1,#2)}

\newcommand{\gruKPGame}[1]{Gru_{K,P}(#1)}
\newcommand{\gruKLGame}[1]{Gru_{K,L}(#1)}

\newcommand{\cloPFGame}[1]{PtFin_{F,C}(#1)}

\newcommand{\menGame}[1]{Men_{C,F}(#1)}
\newcommand{\rothGame}[1]{Roth_{C,S}(#1)}
\newcommand{\rothAltGame}[1]{Roth_{P,O}(#1)}

\newcommand{\bellConGame}[1]{Bell_{D,P}(#1)}



\newcommand{\SigmaProdR}[1]{\Sigma\mathbb{R}^{#1}}
\newcommand{\sigmaprodtwo}[1]{\Sigma2^{#1}}

\newcommand{\concat}{{^\frown}}
\newcommand{\rest}{\restriction}

\newcommand{\cl}[1]{\overline{#1}}

\newcommand{\pow}[1]{\mc{P}(#1)}

\newcommand{\<}{\langle}
\renewcommand{\>}{\rangle}

\newcommand{\al}[1]{{#1}^*}

\newcommand{\mc}[1]{\mathcal{#1}}
\newcommand{\mb}[1]{\mathbb{#1}}

\newcommand{\po}{\mathbb{P}}
\newcommand{\pok}{\po_\kappa}

\newcommand{\Lim}{\mathrm{Lim}}
\newcommand{\Suc}{\mathrm{Suc}}

\newcommand{\ds}{\displaystyle}

\newcommand{\st}[2]{st\left(#1,#2\right)}

\newcommand{\alcomp}{\al\parallel}

\newcommand{\rank}{\textrm{rank}}
\newcommand{\dom}{\textrm{dom}}

\renewcommand{\mod}{\,\textrm{mod}}

\newcommand{\zip}{\bowtie}
\newcommand{\ran}[1]{\text{range}(#1)}

\newcommand{\cf}[1]{\textrm{cf}(#1)}

\newcommand{\alcompS}[1]{S(#1,\omega,\omega)}


\newcommand{\scish}{almost-$\sigma$-(relatively compact)}

\usepackage{mathrsfs}
\newcommand{\pl}[1]{\mathscr{#1}}



\newcommand{\term}{\textit}


\newcommand{\bakerGame}[1]{{Bak}_{A,B}(#1)}
\newcommand{\bmGame}[1]{{BM}_{E,N}(#1)}



%%%%%%%%%%%%
% Document %
%%%%%%%%%%%%


\begin{document}

\begin{center}

\textsc{\huge Steven Clontz}

% Title
\HRule \\[0.1cm]
{ \huge \bfseries Teaching Statement \\[0.4cm] }

\HRule \\[1.5cm]

\end{center}

My philosophy of teaching is based upon the assumption that  I am lucky enough to have a full-time job studying what I think is the  most interesting subject in the world. If this assumption is valid, then it only makes sense that my students should be able to enjoy the pursuit of mathematical understanding as much as I do. To this end, I have spent much of my time at South honing my skills as an educator, taking advantage of the training provided by the Innovations in Learning Center, using the expertise of my colleauges, and translating  my personal experiences as a game and puzzle designer into my course design. 

Before my appointment at South, I had experience using active learning, through both inquiry-based learning and group-based classrooms. To this end, I hit the ground running with a flipped-style Calculus II classroom where students would watch my lectures online, and spend class time collaborating in groups on related exercises. The efficacy of flipped learning in mathematics is witnessed in the literature; see [0]. While I observed some gains compared with a traditional lecture, I felt that students were not being challenged to go beyond rote memorization and repetition of the techniques I demonstrated in my videos. To learn more about how I could motivate my students to reach higher levels of Bloom's Taxonomy, I turned to the Innovations in Learning Center and TeamUSA. 

TeamUSA is our accreditation-required Quality Enhancement Plan at South, which trains faculty to use a form of active learning called Team-Based Learning (TBL). According to TeamUSA's website, the use of TBL facilitiates the improvement of student learning outcomes, the achievement of higher levels of critical thinking  and problem-solving skills, the enhancement of collaboration and communication, and the application of course content to real-world situations. However, there has been sparce research done specifically in mathematics on the use of TBL, particuarly in comparison to inquiry-based learning, where the research indicates improvements over traditional lectures, particularly for women [1].  To this end, I've begun to collaborate with my colleague Drew Lewis to develop Team-Based Inquiry Learning, which aims to marry the benefits of TBL's educational scaffolding with the active mathematical engagement found in an inquiry-based learning classroom. We are currently co-authoring a paper with Julie Estis, the director of TeamUSA, on our implementation of TBIL in sophomore linear algebra courses at South. 

In addition to these academic endeavors, I also have the benefit of my experience as a puzzle and game designer. There's no greater incentive to persevere when struggling with difficult mathematical ideas than when the struggle itself a fun exercise. Team-Based Learning helps with this a lot, as my students are not isolated in their endeavor, but can rely on their peers to bounce around ideas and consult directly with myself as needed. But just as importantly, this struggle has to be approriately challenging; players/students will abandon a puzzle that seems too hard or even unfair. This is where my experience in designing fun puzzle events such as the Mathematical Puzzle Programs (MaPP) Challenge for grade school students or the Dimensons Puzzlehunt at the National Museum of Mathematics comes into play: I go through great lengths to ensure that my classroom activities are appropriately scaffolded to keep students progressing through the material and discovering as much of the mathematics as they can for themselves.  

Furthermore, due to the implementation of Mastery Grading in my courses, students are never truly left behind.   I am an active member of the Mastery Grading academic community, participating both on social media and in workshops and sessions such as the one held during the Mathematical Association of America's 2018 annual meeting. Using mastery grading, students are given a list of standards, mathematical ideas that they need to master by the end of the course. Each quiz is an opportunity to demonstrate that mastery, as usual in most courses. However, rather than awarding partial credit to solutions that demonstrate incomplete understanding, students must either revise  a nearly-correct solution to be completely correct, or must reattempt a new exercise either in my office or on a future assessment. Only when the student provides me with a completely correct solution will this standard be checked off their progress report, and their final letter grade for the course is based purely based upon the number of these standards the student has demonstrated (complete!) mastery over by the end of the course.  This style of grading not only matches my intuition as a game designer who wants to incentivize players/students to continue to persevere with difficult concepts, but it is also supported in the literature, see [2] as an example. The implementation of mastery grading has allowed me to hold my students to a higher standard for success in my courses, without increasing failure rates or losing significant student buy-in.   

[0] Betty Love, Angie Hodge, Neal Grandgenett \& Andrew W. Swift (2014) Student learning and perceptions in a flipped linear algebra course, International Journal of Mathematical Education in Science and Technology, 45:3, 317-324, DOI: 10.1080/0020739X.2013.822582 

[1] Sandra L. Laursen, Marja-Liisa Hassi, Marina Kogan, \& Timothy J. Weston. (2014). Benefits for Women and Men of Inquiry-Based Learning in College Mathematics: A Multi-Institution Study. Journal for Research in Mathematics Education, 45(4), 406-418. doi:10.5951/jresematheduc.45.4.0406

[2] Mike Brilleslyper, Michelle Ghrist, Trae Holcomb, Beth Schaubroeck, Bradley Warner \& Scott Williams (2012) What's The Point? The Benefits of Grading Without Points, PRIMUS, 22:5, 411-427, DOI: 10.1080/10511970.2011.571346


\end{document}
