\documentclass[11pt]{amsart}
\usepackage{fancyhdr}
\usepackage[margin=1in]{geometry}

\usepackage[page]{totalcount}

\usepackage{amsthm}
\theoremstyle{plain}
\newtheorem{theorem}{Theorem}
\newtheorem{question}{Question}

\usepackage[pdfpagelabels]{hyperref}
\hypersetup{colorlinks=true,urlcolor=blue,linkcolor=black,citecolor=black}

% \usepackage{fontspec}
%   \defaultfontfeatures{Ligatures=TeX}
%   \setmainfont{Times New Roman}
% \usepackage{setspace}
% \doublespacing

\pagestyle{fancy} \headheight 15pt \footskip 20pt

\parskip=12pt

\rhead{\thepage/\totalpages}
\chead{}
\lhead{Clontz Teaching Statement}
\rfoot{\today}
\cfoot{}
\lfoot{}

\newcommand{\HRule}{\rule{\linewidth}{0.5mm}}


%%%%%%%%%%%%%%%
% Definitions %
%%%%%%%%%%%%%%%

% Strategy uparrow shortcuts
\newcommand{\win}{\uparrow}
\newcommand{\prewin}{\underset{\text{pre}}{\uparrow}}
\newcommand{\markwin}{\underset{\text{mark}}{\uparrow}}
\newcommand{\tactwin}{\underset{\text{tact}}{\uparrow}}
\newcommand{\kmarkwin}[1]{\underset{#1\text{-mark}}{\uparrow}}
\newcommand{\ktactwin}[1]{\underset{#1\text{-tact}}{\uparrow}}
\newcommand{\codewin}{\underset{\text{code}}{\uparrow}}
\newcommand{\limitwin}{\underset{\text{limit}}{\uparrow}}
\newcommand{\notwin}{\not\uparrow}
\newcommand{\notprewin}{\underset{\text{pre}}{\not\uparrow}}
\newcommand{\notmarkwin}{\underset{\text{mark}}{\not\uparrow}}
\newcommand{\nottactwin}{\underset{\text{tact}}{\not\uparrow}}
\newcommand{\notkmarkwin}[1]{\underset{#1\text{-mark}}{\not\uparrow}}
\newcommand{\notktactwin}[1]{\underset{#1\text{-tact}}{\not\uparrow}}
\newcommand{\notcodewin}{\underset{\text{code}}{\not\uparrow}}
\newcommand{\notlimitwin}{\underset{\text{limit}}{\not\uparrow}}

\newcommand{\oneptcomp}[1]{#1^*}
\newcommand{\oneptlind}[1]{#1^\dagger}
% \newcommand{\sharp}[1]{#1^{\#}}

% Games
\newcommand{\gruConGame}[2]{Gru_{O,P}(#1,#2)}

\newcommand{\gruKPGame}[1]{Gru_{K,P}(#1)}
\newcommand{\gruKLGame}[1]{Gru_{K,L}(#1)}

\newcommand{\cloPFGame}[1]{PtFin_{F,C}(#1)}

\newcommand{\menGame}[1]{Men_{C,F}(#1)}
\newcommand{\rothGame}[1]{Roth_{C,S}(#1)}
\newcommand{\rothAltGame}[1]{Roth_{P,O}(#1)}

\newcommand{\bellConGame}[1]{Bell_{D,P}(#1)}



\newcommand{\SigmaProdR}[1]{\Sigma\mathbb{R}^{#1}}
\newcommand{\sigmaprodtwo}[1]{\Sigma2^{#1}}

\newcommand{\concat}{{^\frown}}
\newcommand{\rest}{\restriction}

\newcommand{\cl}[1]{\overline{#1}}

\newcommand{\pow}[1]{\mc{P}(#1)}

\newcommand{\<}{\langle}
\renewcommand{\>}{\rangle}

\newcommand{\al}[1]{{#1}^*}

\newcommand{\mc}[1]{\mathcal{#1}}
\newcommand{\mb}[1]{\mathbb{#1}}

\newcommand{\po}{\mathbb{P}}
\newcommand{\pok}{\po_\kappa}

\newcommand{\Lim}{\mathrm{Lim}}
\newcommand{\Suc}{\mathrm{Suc}}

\newcommand{\ds}{\displaystyle}

\newcommand{\st}[2]{st\left(#1,#2\right)}

\newcommand{\alcomp}{\al\parallel}

\newcommand{\rank}{\textrm{rank}}
\newcommand{\dom}{\textrm{dom}}

\renewcommand{\mod}{\,\textrm{mod}}

\newcommand{\zip}{\bowtie}
\newcommand{\ran}[1]{\text{range}(#1)}

\newcommand{\cf}[1]{\textrm{cf}(#1)}

\newcommand{\alcompS}[1]{S(#1,\omega,\omega)}


\newcommand{\scish}{almost-$\sigma$-(relatively compact)}

\usepackage{mathrsfs}
\newcommand{\pl}[1]{\mathscr{#1}}



\newcommand{\term}{\textit}


\newcommand{\bakerGame}[1]{{Bak}_{A,B}(#1)}
\newcommand{\bmGame}[1]{{BM}_{E,N}(#1)}



%%%%%%%%%%%%
% Document %
%%%%%%%%%%%%


\begin{document}

\begin{center}

\textsc{\huge Steven Clontz}

% Title
\HRule \\[0.1cm]
{ \huge \bfseries Teaching and Outreach Statement \\[0.4cm] }

\HRule \\[1.5cm]

\end{center}

Learning should be fun. That wasn't always my experience as a student, but
as an educator it's something I strive for in my classroom.
My first teaching ``job'' was
for a Boy Scout leadership training camp in Florence, AL, beginning when I was
fifteen years old. All instructors at these camps are young men and women
between the ages of fourteen and twenty, and the students range from thirteen to
seventeen years old. During each week-long course, we'd train these youths to
become leaders in their home scouting units through a sequence of lectures,
challenges, and games.

I am proud of the holistic approach to leadership education our small
group developed over the decade I was involved in the program. It wouldn't be
for many years that I would learn jargon such as ``active learning'',
``gamification''', and ``inquiry-based learning'', but we essentially used
variations on all these techniques to teach our students how to be teachers
and leaders, both in scouting and in the rest of their lives.
While mathematics education is distinct from leadership training in many ways,
I've been thankful that I have had the opportunity to spend this large portion
of my life developing my philosophy of education and putting it into practice.


\section*{Applications of Game and Puzzle Design to Education}

I've always had an interest in games. I spent many hours in my younger years
playing Nintendo and board games, and my parents inform me that my first
answer to ``What do you want to be when you grow up?'' was ``A game show host.''
The core of \textit{The Price is Right} is mainly made up of
rather boring trivia questions about blenders and laundry detergent
intertwined with simple puzzles and games, but
thanks to the Bob Barker and his production staff, the highlight of any
sick day was on CBS between 10am and 11am.
Of course, I eventually realized my true calling was in mathematics and
teaching (and Drew Carey has quite a few years ahead of him on the show,
I'm sure), but as a mathematics educator I use the same skills that
are needed for good game and puzzle design.

A common pitfall that beginner puzzle designers encounter is the desire to
show off their own cleverness by making purposely obfuscated codes or riddles
with several dead ends. Alternately, the designer may neglect to take time to
playtest the puzzle for mistakes and unintended difficulty. These same
shortcomings are found in the assessment design (or lack thereof) in too many
mathematics classrooms. When I create challenges for my students (e.g. homework,
exams, projects), I avoid designing them simply for the purpose of
adversity or grade differentiation. Instead, my assessments are designed to
complement the lessons and skills established within the course for the purpose
of providing students feedback on their own learning.

Any game designer can tell you that polishing individual game mechanics is not
sufficient to create a great game. Similarly, it's not just assessments which
require thoughtful design for a successful classroom.
After all, many students wouldn't take a large final
project seriously if it's only worth 5\% of their overall grade, regardless
of its design. Designing the
course syllabus is treated as an afterthought by too many instructors, when
it's possibly the most important aspect of garnering engagement from students
who are not self-motivated to take part in the class. I take care to
balance my grading system for each course so that the class won't lose
morale or focus from a fear of failing. At the same time I won't allow a
course to be too easy, as even responsible students may then
be tempted to ignore the material in order to
concentrate on other classes or extracurriculars. Combined with assessments
which are designed to promote learning rather than as a chore, each
syllabus I write provides the structure for a class which students want
to and can succeed in.

Returning to the
\textit{Price is Right} analogy, the Plinko minigame would not be as exciting
to watch without the flashing lights and catchy music. Since your students
won't take your course any more seriously than you do, a little production value
can go a long way.  I use many tools in
my classroom to provide a professional and well-designed experience for my
students. I typeset all printed documents in \LaTeX{} and post them online
in PDF format when appropriate. My web development experience allows me to
create tools for my students such as \url{http://gradecalc.stevenclontz.com}.
I enhance my classes by writing code or using tools like Geogebra to create
visual aids for understanding material, and I use Learning Management Systems
like Canvas to keep my students plugged in to the class from wherever
they can log in. In addition to showing my students
that I'm willing to make an extra effort, I believe that using multimedia in
this way brings mathematics to life much better than can be done with just
chalk on a blackboard.


\section*{Active and Inquiry-Based Learning}

As they say, ``Mathematics is not a spectator sport.'' Only a rare student
would be inspired to pursue mathematics just because they listened to someone
lecture about it. Rather, the beauty and fun of mathematics is trying
to solve these puzzles on your own.
As a high school student, I was grateful to be given freedom by my math
teachers to ignore slow-paced lessons to work ahead or on my own
mathematical puzzles. While I craved more engaging lessons at the high school
level, I know now that it's impossible to reach students of all
levels with a single lecture. Even worse, students who fall behind in
mathematics find themselves bored and frustrated with lectures which assume
knowledge they have yet to master. So, how can we broaden our reach within the
mathematics classroom?

In the language of Bloom's taxonomy, one
may characterize learning at six levels: from lowest to highest these are
Remembering, Understanding, Applying, Analyzing, Evaluating, and Creating.
A shortcoming of passive lecturing is the inability to teach or assess
higher order levels of learning: we are only requiring our students to Remember.
However, students in all fields can benefit from real mathematical thinking,
using math to Analyze the world around us, Evaluate situations, and Create
mathematical models or even new mathematical knowledge.
One solution to this problem is to shift the responsibility of learning to
the students themselves by way of active learning. Such a shift allows us to
have our classrooms better emulate the real world of industry and research,
where our students will not always have an expert to rely on for all the
answers.

I've had much success with inquiry-based active learning as both a student
and instructor.
The curriculum of an inquiry-based learning class is carefully organized into
a ``theorem sequence'' of definitions, theorems, and questions, often given as
a whole to the students on the first day of class.
The instructor serves as a coach and moderator, while the students discover
and present proofs and solutions to the theorems and questions asked of them.
Since different proofs and solutions will have different difficulties, this
technique allows students of all abilities to stay active in the course and
work on problems at their own skill level. This method is not limited to
upper-level mathematics courses, and my calculus students have also expressed
appreciation to have an active hand in the time we spend in class and to gain
a deeper understanding of the material.


\section*{Online Mathematics Education}

While face-to-face instruction is usually preferable, online instruction has
a place as an economical solution for covering service courses where resources
are otherwise insufficient. I have experience with the ALEKS learning management
system for teaching hybrid online/lecture courses for the developmental
mathematics curriculum at the community college level.

One advantage to such online systems is the automatic customization of course
material to the ability of each student. In ALEKS, students are assessed at the
beginning of the course and periodically afterwards so that each student is
presented with course topics fitting his or her level of mastery. Thus students
are allowed to move at their own pace, provided the course is completed by
the end of the semester. As an instructor, I am able to use valuable class time
to work one-on-one with individual students as needed, rather than give
a lecture which may not fit the needs of every student.

I have also begun work on my own open-source learning management system: ALMS.
This web application assists in management of an active or inquiry-based
learning classroom by allowing students to work in small groups to
work on custom problem/theorem sets using the simple Markdown markup
language with \LaTeX{} typesetting for mathematical equations.


\section*{Mathematics Outreach}

The work I've done to promote mathematics extends beyond the classroom. My
gateway to mathematics outreach actually began outside the university as a part
of Auburn, Alabama's puzzle competition community. Through organizing and
participating in a series of puzzlehunts and alternate reality games, I
developed many of the game mechanics that would go on to make up the
Auburn Mathematical Puzzle (A.M.P.'d) Challenge for middle school students.

The A.M.P.'d Challenge was co-created by myself and a fellow AU graduate
student with the support of the AU College of Science and Mathematics Office
of Outreach. Breaking away from traditional ciphering and examination-based
competitions, the A.M.P.'d Challenge focuses on inquiry-based active learning
in a team setting. The problems we present are often pulled from the
undergraduate curriculum, such as graph theory or number theory puzzles.
Focusing on problems outside the typical middle school curriculum allows us
to challenge the students on their mathematical problem-solving ability rather
than their recollection of mathematical knowledge from their classrooms.
In addition, the students are given the opportunity to expand their leadership
ability, as they must delegate different puzzles to different players and
put their heads together to overcome new problems. Due to its success,
I was hired by Lamar University to bring this sort of experience to a high
school audience in 2015, and I continue to serve as a consultant for the
Auburn Univeristy A.M.P.'d Challenge. I am working with several colleagues
across the country to pivot Lamar Mathematical Puzzle Challenge into
a national multi-campus event in 2016.

Beyond being fun for both the participants and myself, these mathematical
outreach programs have given me the opportunity to connect with my
community and others. Too often I've found that the post-secondary mathematics
classroom is divorced from the current experience of high school mathematics.
Organizing such events gives me a unique
perspective on secondary education, which better enables me as an educator of
undergraduate mathematics students.


\section*{Teaching Evaluations}

Teaching evaluations for many of the courses I've taught at Auburn University
are available on my website: \url{http://stevenclontz.com/teaching/}.

\vfill


\end{document}