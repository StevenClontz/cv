\documentclass[11pt]{amsart}
\usepackage{fancyhdr}
\usepackage[margin=1in]{geometry}

\usepackage[page]{totalcount}

\usepackage{amsthm}
\theoremstyle{plain}
\newtheorem{theorem}{Theorem}
\newtheorem{question}{Question}

\usepackage[pdfpagelabels]{hyperref}
\hypersetup{colorlinks=true,urlcolor=blue,linkcolor=black,citecolor=black}

% \usepackage{fontspec}
% 	\defaultfontfeatures{Ligatures=TeX}
% 	\setmainfont{Times New Roman}
% \usepackage{setspace}
% \doublespacing

\pagestyle{fancy} \headheight 15pt \footskip 20pt

\parskip=12pt

\rhead{\thepage/\totalpages}
\chead{}
\lhead{Clontz Research Statement}
\rfoot{\today}
\cfoot{}
\lfoot{}

\newcommand{\HRule}{\rule{\linewidth}{0.5mm}}


%%%%%%%%%%%%%%%
% Definitions %
%%%%%%%%%%%%%%%

% Strategy uparrow shortcuts
\newcommand{\win}{\uparrow}
\newcommand{\markwin}{\underset{\text{mark}}{\uparrow}}
\newcommand{\tactwin}{\underset{\text{tact}}{\uparrow}}
\newcommand{\kmarkwin}[1]{\underset{#1\text{-mark}}{\uparrow}}
\newcommand{\ktactwin}[1]{\underset{#1\text{-tact}}{\uparrow}}
\newcommand{\codewin}{\underset{\text{code}}{\uparrow}}
\newcommand{\limitwin}{\underset{\text{limit}}{\uparrow}}
\newcommand{\notwin}{\not\uparrow}
\newcommand{\notmarkwin}{\underset{\text{mark}}{\not\uparrow}}
\newcommand{\nottactwin}{\underset{\text{tact}}{\not\uparrow}}
\newcommand{\notkmarkwin}[1]{\underset{#1\text{-mark}}{\not\uparrow}}
\newcommand{\notktactwin}[1]{\underset{#1\text{-tact}}{\not\uparrow}}
\newcommand{\notcodewin}{\underset{\text{code}}{\not\uparrow}}
\newcommand{\notlimitwin}{\underset{\text{limit}}{\not\uparrow}}

\newcommand{\oneptcomp}[1]{#1^\star}
\newcommand{\oneptlind}[1]{#1^\dagger}
% \newcommand{\sharp}[1]{#1^{\#}}

% Games
\newcommand{\gruConGame}[2]{Gru_{O,P}(#1,#2)}

\newcommand{\gruKPGame}[1]{Gru_{K,P}(#1)}
\newcommand{\gruKLGame}[1]{Gru_{K,L}(#1)}

\newcommand{\cloPFGame}[1]{PtFin_{F,C}(#1)}

\newcommand{\menGame}[1]{Men_{C,F}(#1)}
\newcommand{\rothGame}[1]{Roth_{C,S}(#1)}
\newcommand{\rothAltGame}[1]{Roth_{P,O}(#1)}

\newcommand{\bellConGame}[1]{Bell_{D,P}(#1)}



\newcommand{\SigmaProdR}[1]{\Sigma\mathbb{R}^{#1}}
\newcommand{\sigmaprodtwo}[1]{\Sigma2^{#1}}

\newcommand{\concat}{{^\frown}}
\newcommand{\rest}{\restriction}

\newcommand{\cl}[1]{\overline{#1}}

\newcommand{\pow}[1]{\mc{P}(#1)}

\newcommand{\<}{\langle}
\renewcommand{\>}{\rangle}

\newcommand{\al}[1]{{#1}^*}

\newcommand{\mc}[1]{\mathcal{#1}}
\newcommand{\mb}[1]{\mathbb{#1}}

\newcommand{\po}{\mathbb{P}}
\newcommand{\pok}{\po_\kappa}

\newcommand{\Lim}{\mathrm{Lim}}
\newcommand{\Suc}{\mathrm{Suc}}

\newcommand{\ds}{\displaystyle}

\newcommand{\st}[2]{st\left(#1,#2\right)}

\newcommand{\alcomp}{\al\parallel}

\newcommand{\rank}{\textrm{rank}}
\newcommand{\dom}{\textrm{dom}}

\renewcommand{\mod}{\,\textrm{mod}}

\newcommand{\zip}{\bowtie}
\newcommand{\ran}[1]{\text{range}(#1)}

\newcommand{\cf}[1]{\textrm{cf}(#1)}

\newcommand{\alcompS}[1]{S(#1)}


\newcommand{\scish}{almost-$\sigma$-(relatively compact)}

\usepackage{mathrsfs}
\newcommand{\pl}[1]{\mathscr{#1}}



\newcommand{\term}{\textit}


\newcommand{\bakerGame}[1]{{Bak}_{A,B}(#1)}
\newcommand{\bmGame}[1]{{BM}_{E,N}(#1)}



%%%%%%%%%%%%
% Document %
%%%%%%%%%%%%


\begin{document}

\begin{center}

\textsc{\huge Steven Clontz}

% Title
\HRule \\[0.1cm]
{ \huge \bfseries Research Statement \\[0.4cm] }

\HRule \\[1.5cm]

\end{center}


My emerging career as a mathematical researcher is punctuated with several
results, invited talks, and multiple accepted publications with several
more in perparation. I am passionate not
only about the creation of my research, but
also about the cyberinfrastructure which we use to share and collaborate
as 21st-century mathematicians. My
experience with the open web provides me with a unique perspective on academic
collaboration in the information age, which will be changing drastically
in the coming decade. In addition, I'm excited to apply my history of
leadership training and mentorship in order to guide young mathematicians
in their own research experience, beginning at the undergraduate level.


\section*{Undergraduate Research}

My calling to research began in my undergraduate career. Auburn University
(regrettably, in my opinion) does not require any sort of thesis or capstone
project of its undergraduate mathematics majors. Nonetheless, during my junior
year, I worked on some toy problems suggested by two professors at Auburn:
Phil Zenor in topology, and Andras Bezdek in geometry. While I would ultimately
return to topology for my graduate-level research, I was fascinated with a
very tangible and surprisingly deep question on the unfolding of polyhedral
surfaces into the plane, which eventually became the topic of my undergraduate
thesis written as part of the requirements to be recognized as a University
Honors Scholar \cite{UNDERGRAD}.

Suppose that a polyhedron can be cut along its edges and unfolded into
the plane; we refer to such an unfolding as a \term{net} of the polyhedron.
It is easy to construct polyhedra with non-convex faces which do not
yield a net. A classic example of a polyhedron with convex faces which does
not admit a net involves gluing so-called \term{witch's hats} to the faces
of a regular octahedron; however, this polyhedron is not convex
\cite{UNUNFOLDABLE}.

G. C. Shephard conjectured in 1975 that every convex polyhedron yields at least
one net \cite{MR0390915}. If it is surprising that this question had
not been explicitly posed previously, then it is perhaps more shocking that
such a seemingly simple statement remains open to this day. For a beginning
mathematician, this was an exciting opportunity to engage a research-level
problem which was accessible without graduate-level coursework.
Indeed, the basic premise requires no mathematical background to appreciate,
and I successfully applied and interviewed for undergraduate research
fellowships at the university and college levels.

My thesis investigated a small subclass of convex polyhedra which I denoted
\term{generalized pyramids}: the convex hull of two parallel polygons, one of
which may be degenerate and whose projection must lay within the other.
In addition to a survey of the problem of unfolding polyhedra, I outlined
two techniques for unfolding certain generalized pyramids, based upon
techniques which work for typical pyramids.

Due to my experience as an undergraduate researcher in mathematics, I gained
my initial excitement for discovering new mathematical ideas. But just as
importantly, I also learned a lot about how I would want to mentor other
undergraduate researchers beginning their own mathematical journeys. In my
first year at UNC Charlotte, I've already begun to do so; I have taken
on my first student and will mentor her during Fall 2015
as she writes a survey on finite and infinite combinatiral game theory. We
chose this topic as it relates to my own current research on topological
games, but it also is accessible to both an undergraduate student with a
generalized background and to the layman.


\section*{Current Research}

I remained at Auburn to pursue a fifth-year mathematics education program,
but ultimately committed full-time to mathematics, including coursework in
general topology and combinatorial game theory. Before beginning my doctoral
work, I wrote a masters thesis
under Gary Gruenhage surveying known applications of stationary subsets of
regular uncountable cardinals \cite{MASTERS}.

I was first introduced to infinite combinatorial games during a descriptive
set theory course. Eager to learn more, I translated a paper by G. Debs
\cite{MR817083} constructing a space for which the second player has a winning
strategy in the Banach-Mazur game, but not a winning tactical
strategy. This paper served as the inspiration for my doctoral work at
Auburn under Gary Gruenhage.

\subsection*{Abstract}

The majority of my research involves studying the applications of
limited information strategies in various topological and set-theoretic
games. More recently,
I've also investigated generalized inverse limits with an ordinal-valued
index set. Particular results of note include:
  \begin{itemize}
    \item extending results of Peter Nyikos \cite{MR1031771} on limited
          information strategies in Gruenhage's open-point convergence game,
    \item answering a question of Nyikos \cite{MR3288115} showing
          that Bell's proximal game characterizes the class of Corson compact
          spaces (joint work with Gary Gruenhage \cite{MR3227201})
    \item characterizing strong Eberlein compactness with the existence of
          a tactical winning strategy for Bell's proximal game
          \cite{tacticProximal}
    \item showing an example of a locally compact, non-metacompact space for
          which the first player has a winning strategy but no winning
          $k$-tactical strategy for Gruenhage's compact-point game
          \cite{ktacticsCompactOpen}
    \item extending results of Telgarsky \cite{MR753073} and Scheepers
          \cite{MR1273523} to characterize
          game-theoretic strengthenings of the Menger property and other
          selection properties, and
    \item proving that a generalized invesrse limit with index set \(\kappa\)
          and an idempotent, continuum-valued, surjective u.s.c. bonding map
          is not Corson compact (joint work with Scott Varagona
          \cite{destroyMetrizability}).
  \end{itemize}

% \subsection*{Definitions}

% Intuitively, an
% \term{$\omega$-length game} involves two players exchanging \term{moves} by
% selecting elements from a given set over the course of $\omega=\{0,1,2,\dots\}$
% rounds. At the ``end'' of the game, the $\omega$-length sequence of choices
% made by the two players
% are inspected to see if it fits the game's \term{winning condition} for a
% particular player; if so, that player wins, and if not, the opponent wins.
% Rigorously, this may be modeled by a tuple $\<M,W\>$ where $M$ is the set of
% moves from which the players may choose from, and $W$ is the set of
% $\omega$-length sequences of moves
% for which the first player wins the game, called the \term{winning playthroughs}
% for the first player. Thus $M^\omega\setminus W$ would be the set of winning
% playthroughs for the second player, as no ties are allowed.
% The introduction of such games is commonly attributed to Gale and Stewart
% \cite{MR0054922}.

% A \term{topological
% game} is an $\omega$-length game where the moveset $M$ is related to the
% topological structure of a given space $X$.
% Perhaps the most prolific example of such a game
% is the \term{Banach-Mazur game} $\bmGame{X}$ played on a topological space $X$.
% During each round of this game, each player must choose an open subset of all
% previously played open sets, starting with player $\pl E$.
% Player $\pl E$ wins the game if the intersection
% of all the open sets is empty, and player $\pl N$ wins the game otherwise.

% A function $\sigma:M^{<\omega}\to M$ is known as a \term{strategy} for a game
% with moveset $M$: intuitively, it determines the moves for a player given
% all the previous moves of her opponent. If there exists a strategy $\sigma$
% such that a player $\pl A$ will always win a game $G$ while using it, then
% we say $\sigma$ is a \term{winning strategy} and write $\pl A\win G$
% (``$\pl A$ wins $G$''). (All finite-length games have a player with a
% winning strategy; however, under $ZFC$ this need not
% hold for an infinite game.) The presence or absence of a winning strategy for
% a player in a topological game on $X$ characterizes
% a topological property of $X$. For example, a space $X$ is \term{Baire}
% (the countable intersection of open dense sets is dense) if and only if
% $\pl E\notwin \bmGame{X}$.

% Sometimes a player may not need to use the complete history of her opponent's
% moves to win a game. We call such strategies
% \term{limited information strategies}. A strategy
% which only uses the latest move (last $k$ moves) of the opponent is called
% a \term{tactical} (\term{$k$-tactical}) strategy, and if it is winning we
% write $\pl A\tactwin G$ ($\pl A\ktactwin{k} G$). A strategy
% which only uses the latest move (last $k$ moves, no moves) of the opponent
% and the number of the current round is called a \term{Markov} (\term{$k$-Markov},
% \term{predetermined}) strategy, and if it is winning we
% write $\pl A\markwin G$ ($\pl A\kmarkwin{k} G$, $\pl A\kmarkwin0 G$). A strategy
% which uses the latest move of both players is called a \term{coding} strategy,
% and if it is winning we write $\pl A\codewin G$. The presence or absence
% of a winning limited information strategy for a player in a topological game
% on $X$ may characterize a stronger property of $X$ than the one characterized
% by a winning perfect information strategy. For example, $\pl N \win \bmGame X$
% if and only if $\pl N \codewin \bmGame X$; however, $\pl N \tactwin \bmGame X$
% if and only if $X$ is a \term{siftable} space (all siftable spaces are thus
% Baire, but the converse does not hold). A famous question asks if for each
% $k<\omega$ there exists $X_k$ such that $\pl N\ktactwin{k+1}\bmGame{X_k}$
% but $\pl N\notktactwin{k}\bmGame{X_k}$. (The non-trivial example of Debs
% referenced earlier witnesses $k=1$.)

% For a more complete overview of the history of topological games, I recommend
% R. Telgarsky's excellent survey dedicated to the Banach Mazur game's 50th
% anniversary \cite{MR892457}.

\subsection*{Results and Open Questions}

G. Gruenhage introduced
the $\omega$-length game $\gruConGame{X}{x}$ where during
each round the first player $\pl O$ chooses a neighborhood of $x$ in $X$, and
the second player $\pl P$ chooses a point within all the previously chosen
neighborhoods \cite{MR0413049}.
$\pl O$ wins this game if the points chosen by $\pl P$
converge to $x$. It's easy to see that if $x$ has a countable base,
then $\pl O \kmarkwin0 \gruConGame{X}{x}$, that is, $\pl O$ has a winning
$0$-Markov strategy using only knowledge of the round number; this in fact is
a game-theoretic characterization of first-countability at $x$. The well-known
class of $W$ spaces are the spaces for which $\pl O \win \gruConGame{X}{x}$
($\pl O$ has a winning strategy using full information)
for all $x\in X$; all first-countable spaces are then $W$ spaces.

To investigate the limited-information setting for $\gruConGame{X}{x}$, we look
to non-first-countable $W$ spaces. Examples which I have studied include the
one-point compactification of a discrete cardinal
$\oneptcomp\kappa=\kappa\cup\{\infty\}$ and the
$\Sigma$-product of $\kappa$ real lines $\Sigma\mb R^\kappa$. (Note that
$\oneptcomp\kappa$ is a subspace of $\Sigma\mb R^\kappa$ where
$\infty\in\oneptcomp\kappa$ corresponds to $\vec0\in\Sigma\mb R^\kappa$.)

P. Nyikos first showed that
$\pl O\notmarkwin\gruConGame{\oneptcomp\omega_1}{\infty}$ when $\pl P$ is only
required to play within the latest open set played \cite{MR1031771}.
When $\pl P$ must play within all previously played open sets, I have shown
that $\pl O\notkmarkwin{k}\gruConGame{\oneptcomp\kappa}{\infty}$ for
$\kappa>\omega_1$ and
$\pl O\notktactwin{k}\gruConGame{\oneptcomp\omega_1}{\infty}$, but knowledge
of the round number allows
$\pl O\markwin\gruConGame{\oneptcomp\omega_1}{\infty}$. When $\pl O$
is allowed to use her most recent move as well as her opponents,
she may win by encoding information on the history of the game into
her own moves, that is,
$\pl O\codewin\gruConGame{\Sigma\mb R^\kappa}{\vec0}$ for all cardinals
$\kappa$.

This result suggests an analogous question to those asked of the
Banach-Mazur game.

\begin{question}
  Does $\pl O\codewin\gruConGame{X}{x}$ imply $\pl O\win\gruConGame{X}{x}$?
\end{question}

Recently, J. Bell introduced a related game $\bellConGame{X}$ for uniformizable
spaces \cite{MR3239205}. During each round of this game, $\pl D$ chooses an
entourage of the diagonal, and $\pl P$ chooses a point within the previous
entourage of the previous point.
$\pl D$ wins this game if either the points chosen by $\pl P$ converge,
or the intersection of the entourage neighborhoods have empty intersection.
Spaces for which $\pl D$ has a winning strategy
are said to be \term{proximal}.

Bell demonstrated that all proximal spaces
are $W$ spaces; I've extended this by showing that for any $k<\omega$,
a winning $2k$-Markov (resp. $2k$-tactical) strategy for Bell's game can be
used to construct a winning $k$-Markov (resp. $k$-tactical) strategy for
$\gruConGame{X}{x}$. In addition, if $X$ only has a single non-isolated point,
then a winning $k$-Markov (resp. $k$-tactical) strategy in either game can be
used to construct a winning $k$-Markov (resp. $k$-tactical) strategy in the
other game.

Compact subspaces of $\Sigma$-products of real lines are called
\term{Corson compact}. Since $\Sigma$-products of proximal spaces and
closed subspaces of proximal spaces are proximal, all Corson compacts are
easily seen to be proximal. I showed with Gruenhage in \cite{MR3227201} that
the converse also holds. To see this, we gave a non-trivial proof that when
$\pl D\win\bellConGame{X}$ and $X$ is compact, then
$\pl O\win\gruConGame{X}{H}$ for any closed subset
$H$ of $X$. This yields the desired result when combined with a lemma due to
Gruenhage in \cite{MR752278}: a
compact space is Corson compact if and only if
$\pl O\win\gruConGame{X^2}{\Delta}$, where $\Delta$ is the diagonal of $X^2$.

Since $\pl O\markwin\gruConGame{X^2}{\Delta}$ characterizes \term{Eberlein
compactness} in the category of compact spaces, and
$\pl D\markwin\bellConGame{X}$ for any Eberlein compact space,
a natural question arises
concerning the relationship between Bell's game and Gruenhage's game.

\begin{question}
  Does $\pl D\markwin\bellConGame{X}$
  characterize Eberlein compactness in the category of compact spaces?
\end{question}

\begin{question}
  Does $\pl D\win\bellConGame{X}$ characterize all closed subspaces of
  a $\Sigma$-product of real lines (as posed by P. Nyikos)?
\end{question}

More recently, I have shown that $\pl O\tactwin\bellConGame{X}$ characterizes
\term{strong Eberlein compactness} \cite{tacticProximal}. This has shown
that $\bellConGame{X}$ and $\gruConGame{X^2}{\Delta}$ are not equivalent
games for compact spaces when considering limited information:
$\pl D\tactwin\gruConGame{X^2}{\Delta}$ for any compact metrizable space,
but many
compact metrizable spaces (e.g. $I=[0,1]$) are not strong Eberlein compact.

Another game related to $\gruConGame{X}{x}$ is $\gruKPGame{X}$. During each
round of this game, $\pl K$ chooses a compact set in $X$, followed by
$\pl P$ choosing a point outside every previously chosen compact set. $\pl K$
wins this game if the points chosen by $\pl P$ are locally finite in the
space. In the case that $X$ is locally compact, this is essentially the
same game as $\gruConGame{\oneptcomp X}{\infty}$ where neighborhoods of
$\infty$ in $\oneptcomp X=X\cup\{\infty\}$ are complements of compact sets
in $X$.

Gruenhage showed in \cite{MR858337} that for locally compact spaces,
$\pl K\tactwin\gruKPGame{X}$ if and
only if $X$ is metacompact, and $\pl K\markwin\gruKPGame{X}$ if and only if
$X$ is $\sigma$-metacompact. I've extended this to show that for locally
compact or Hausdorff compactly-generated spaces,
$\pl K\kmarkwin0\gruKPGame{X}$ if and only if $X$ is hemicompact.

In fact, for
locally compact or Hausdorff compactly-generated spaces, this game is
equivalent with respect to predetermined strategies to a variation
$\gruKLGame{X}$ where the second player is able to choose compact sets
instead of points. However, there exists an ultrafilter
$\mc F\in\beta\omega\setminus\omega$ such that the single-ultrafilter space
$\omega\cup\{\mc F\}$
is an example of a Hausdorff non-compactly-generated space with
$\pl K\kmarkwin0\gruKPGame{X}$ but $\pl K\notkmarkwin0\gruKLGame{X}$. Due to a lack
of counter-examples, it's natural to ask:

\begin{question}
  Does $\pl K\kmarkwin0\gruKLGame{X}$ imply $X$ is compactly generated?
\end{question}

Gruenhage suggested a consistent example of a locally compact non-metacompact
space for which $\pl K$ has a perfect information strategy which may allow
a winning $2$-tactical strategy for $\pl K$. However, I have shown that this
space does not allow a winning $k$-tactical strategy for any $k<\omega$, which
leaves the following question.

\begin{question}
  For locally compact spaces,
  does $\pl K\ktactwin{2}\gruKPGame{X}$ imply
  $X$ is metacompact?
\end{question}

I have investigated the classic topological game $\menGame{X}$
characterizing the \term{Menger} property, in addition to several other
selection property games of the form $G_{fin}(\mc A,\mc B)$ as studied
by Scheepers \cite{MR1378387} and others.
During round $n$ of such games, $\pl A$ chooses an element $A_n$ of $\mc A$,
and $\pl B$ chooses some finite subset of $A_n$.
$\pl B$ wins if the union of the chosen
subsets is a member of $\mc B$. In this language, $\menGame{X}$ is the special
case where $\mc A=\mc B=\mc O$, the collection of open covers of $X$,
and denoting $\pl A$ by $\pl C$ and $\pl B$ by $\pl F$.

Telgarsky \cite{MR753073} and Scheepers \cite{MR1273523} provided proofs of
the fact that for metrizable spaces, $\pl F\win\menGame{X}$
characterizes $\sigma$-compactness. I have broken this down to show that
$\pl F\markwin\menGame{X}$ characterizes $\sigma$-compactness for regular
spaces, and a winning perfect information strategy for a second-countable
space may be improved to a winning Markov strategy.

Since a winning $(k+2)$-Markov strategy can always be used to construct a
winning $2$-Markov strategy, I have investigated the class of
non-$\sigma$-compact spaces for which $\pl F\kmarkwin{2}\menGame{X}$.
The one-point Lindel\"ofication $\oneptlind\omega_1=\omega_1\cup\{\infty\}$
of a discrete first-uncountable space (neighborhoods of $\infty$ have countable
complements) is a $ZFC$ example. Assuming an axiom $\alcompS\kappa$
independent of $ZFC$ for $\omega_1<\kappa\leq 2^\omega$
and introduced by Scheepers in \cite{MR1129143}
to study a set-theoretic game similar to $\menGame{\oneptlind\kappa}$,
I have shown that $\pl F\kmarkwin{2}\menGame{\oneptlind\kappa}$. Several
games similar to $\menGame{\oneptlind\kappa}$ can be won with a $2$-Markov
or $2$-tactical strategy assuming $\alcompS\kappa$, so it would be interesting
to construct a winning strategy when $\alcompS\kappa$ fails.

\begin{question}
  When does there exist a cardinal $\kappa$ such that
  $\pl F\kmarkwin{2}\menGame{\oneptlind\kappa}$ but $\neg\alcompS\kappa$?
\end{question}

It also hasn't been shown if there exists a space which actually requires
perfect information for $\pl F$ to win $\menGame{X}$.

\begin{question}
  Is there a space $X$ with $\pl F\win\menGame{X}$ but
  $\pl F\notkmarkwin{2}\menGame{X}$?
\end{question}

Recent developments in continuum theory have involved the study of generalized
inverse limits with set-valued bonding maps and linearly ordered index sets:
$\varprojlim \{X_i,f_{ij},L\}\subseteq \prod_{i\in L}X_i$ satisfying
$x(i)\in f_{ij}(x(j))$ for all $i<j$ in $L$. Since $L$ need not
be countable, metrizability of the space is not guaranteed, and many
other questions in general topology may arise.

I showed
with Scott Varagona in \cite{destroyMetrizability} that when
$X_i=[0,1]$ and $f_{ij}=f$ for some surjective,
idempotent, upper-semicontinuous
$f:[0,1]\to C([0,1])$ (where $C([0,1])$ is the collection of all subcontinua
of $[0,1]$) distinct from the trivial map,
the graph of $f$ must satisfy the so-called condition
$\Gamma$: there exist $x,y$ with $\<x,x\>$, $\<y,y\>$, and $\<x,y\>$ on
the graph of $f$.
It's not hard to then see that when $L$ is an uncountable ordinal,
the inverse limit must then contain a copy of $\omega_1+1$, which
is not metrizable or even Corson compact.

\begin{question}
  Do all compactum-valued, idempotent, surjective, u.s.c. $f:[0,1]\to 2^{[0,1]}$
  satsify condition $\Gamma$?
\end{question}

\begin{question}
  Does there exist a nontrivial idempotent map $f$ and uncountable index $L$
  such that $\varprojlim \{[0,1],f,L\}$ is metrizable?
\end{question}


\section*{Mathematical Research and Cyberinfrastructure in the 21st Century}

In addition to researching mathematics, I am also interested in the mechanics
of how we research and collaborate as mathematicians in the 21st century.
While we have made progress in recent years as journals move to make their
archives available online and sites like arXiv.org provide convenient platforms
for sharing preprints of cutting-edge results, we are still
inefficiently passing around virtual sheets of paper.

Mathematics research currently operates as a waterfall process: results
are found, results are written up in a paper (and possibly shared in a
preprint), this paper is submitted for publication, and following the refereeing
and copyediting process, the paper may be published and accepted by the
community. This workflow moves in one direction, and results in a static
product (occasionally with errors).

Thanks to the open web, we have the opportunity to adopt a more agile
workflow. This process is perhaps most
easily seen by looking at collaboration on open source software.
Social coding platforms like GitHub and BitBucket have given software
developers the ability to self-publish cutting-edge software projects for
review by and possible collaboration with the open source community. These
projects are tracked using revision control, with a public record of
contributions made to each project. The process of creating, reviewing, and
publishing such code is a continuous and democratic process, and results
in living projects which can be refined and expanded over time.
By taking advatnage of the tools provided
by the modern web not only to assist with our research, but also with
the publication and collaboration of our research, we will be able to most
efficiently produce and share mathematical knowledge as we move forward in
the modern era.

Projects such as the \textit{$\pi$-Base Topology Database}
(to which I am a contributor and consultant) and
\textit{Homotopy Type Theory: Univalent Foundations of Mathematics} \cite{hottbook}
(licensed under Creative Commons and accepting pull requests on GitHub)
are great examples of how mathematicians can collaborate in producing
high quality resources to assist researchers and students. Likewise, I
am actively developing \textit{Online Seminars in Mathematics},
an initiative to
assist mathematical seminars in streaming their content online so that
researchers may be engaged in such scholarly activity regardless of their
geographical location.
Through these programs and others,
I am excited to continue innovating in this area
as a mathematician and faculty member.




\bibliographystyle{plain}
\bibliography{bibliography}

\end{document}