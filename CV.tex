\documentclass{article}
\usepackage[margin=1in]{geometry}
\usepackage{multicol}
\usepackage{float}
\usepackage{fancyhdr}
\usepackage{enumitem}
\usepackage{amssymb}

\usepackage[page]{totalcount}

\usepackage[colorlinks=true,urlcolor=blue]{hyperref}

\pagestyle{fancy} %\headheight 14.49998pt

\rhead{Clontz \thepage/\totalpages}
\chead{}
\lhead{Curriculum Vitae}
\rfoot{\today}
\cfoot{}
\lfoot{}

\newcommand{\HRule}{\rule{\linewidth}{0.5mm}}

\newcommand{\headerText}[1]{
  \noindent{\large \bf #1} \\*[-.8pc]
  \rule{\textwidth}{.1pt}}

\begin{document}

\thispagestyle{empty}





\begin{center}
{\Large Steven Craig Clontz, Jr.} \\[.5pc]
University of North Carolina at Charlotte |
Department of Mathematics and Statistics \\
(256) 508-3864 \hspace{2em}
  \href{mailto:steven.clontz@gmail.com}{\nolinkurl{steven.clontz@gmail.com}}
  \hspace{2em}
  \url{http://clontz.org}\\
\url{https://github.com/stevenclontz/cv}
\end{center}


\vfill


\headerText{Education} \\
{\bf Doctor of Philosophy, Mathematics,
  Auburn University}\hfill {\it May 2015} \\
Dissertation:
  \textit{Limited information strategies for topological games},
  under Gary Gruenhage\\
GPA: 4.00 \\
Fitzpatrick Fellow in Topology (2012-13, 2014-15),
DMS Teaching Citation (2014-15)\\
\\
{\bf Master of Science, Mathematics,
  Auburn University}\hfill {\it December 2010} \\
Thesis:
  \textit{Applications of stationary sets in set theoretic topology},
  under Gary Gruenhage \\
GPA: 4.00 \\
\\
{\bf Bachelor of Science, Mathematics,
  Auburn University}\hfill {\it May 2008} \\
Honors Thesis:
  \textit{The edge unfolding of generalized pyramids},
  under Andras Bezdek \\
GPA: 3.88\\
Summa Cum Laude, University Honors Scholar, Dean's Medalist,
  Undergraduate Research Fellow \\
Phi Kappa Phi, Phi Beta Kappa\\


\vfill


\headerText{Specialties and Interests}
\begin{multicols}{2}
\begin{itemize}
  \item Set-theoretic topology
  \item Continuum theory
  \item Game theory
  \item Technology in the classroom
  \vfill
  \columnbreak
  \item Cyberinfrastruture of mathematics
        research and education
  \item Mathematical puzzles and games in
        education and outreach
  \item Active and inquiry-based learning
\end{itemize}
\end{multicols}


\vfill


\headerText{Employment and Professional Experience}
\begin{itemize}
  \item
    \textbf{Visiting Assistant Professor},
      University of North Carolina at Charlotte
      (August 2015 - present)
  \item
    \textbf{Director and Founder},
      Mathematical Puzzle Programs
      (May 2015 - present)
  \item
    \textbf{Technology Consultant},
      The National Museum of Mathematics
      (April 2015 - August 2015)
  \item
    \textbf{Founder and Software Engineer},
      Teloga LLC
      (August 2011 - July 2015)
  \item
    \textbf{Graduate Teaching Assistant and Instructor},
      Auburn University Department of Mathematics
      (August 2008 - May 2015)
  \item
    \textbf{Specialist IV - Information Technology},
      Auburn University Office of University Writing
      (January 2014 - May 2014)
  \item
    \textbf{Mathematics Instructor},
      Southern Union State Community College
      (August 2013 - May 2014)
\end{itemize}


\vfill
\newpage


\headerText{Papers}
\begin{itemize}
  \item
    \textit{Proximal compact spaces are Corson compact.}
    Topology Appl. 173 (2014), 1–8. (with G. Gruenhage)
  \item
    \textit{Zero-Markov information in topological games.} Ala. J.
    Math. 39 (2015).
  \item
    \textit{Destruction of metrizability in generalized inverse limits.}
    Top. Proc. 48 (2016) pp. 289-297. (with S. Varagona)
  \item
    \textit{On $k$-tactics for Gruenhage's compact-point game.}
    Q\&A in Gen. Topology. (accepted)
  \item
    \textit{Tactic-proximal compact spaces are strong Eberlein compact.}
    (submitted)
  \item
    \textit{Game-theoretic strengthenings of Menger's property.}
    (submitted)
  \item
    \textit{Almost compatible functions and topological games.}
    (submitted)
  \item
    \textit{Metrizability of inverse limits on an arbitrary linear order.}
    (in preparation)
\end{itemize}


\vfill


\headerText{Selected Presentations}
\begin{itemize}
  \item
    \emph{Applications of almost compatible functions for limited information strategies in infinite length games},
    presentation for
      Boise Extravaganza in Set Theory at San Francisco State University
      (June 2015)
  \item
    \emph{IBL and Mathematical Puzzlehunt Competitions},
    presentation for
      Legacy of R.L. Moore Conference at the Univeristy of Texas
      (June 2015, with PJ Couch)
  % \item
  %   \emph{Fun with Menger's game},
  %     Auburn University DMS Graduate Student Colloquium
  %       (April 2015)
  \item
    \emph{Limited information strategies for a topological proximal game},
    presentations for
      the AMS Sectional Mathematics Meeting at the University of
      Alabama at Huntsville (March 2015) and
      the 49th Spring Topology and Dynamics Conference at
      UNC Greensboro (May 2015)
  % \item
  %   \emph{Proximal compact spaces are Corson compact},
  %   presentation for the AMS/MAA Joint Mathematics Meetings at San Antonio, TX
  %     (January 2015)
  % \item
  %   \emph{Using AngularJS with Ruby on Rails},
  %   invited lecture for The Iron Yard (Atlanta) Ruby on Rails course
  %     (September 2014)
  \item
    \emph{Game-theoretic strengthenings of Menger's property},
    presentations for
      the 29th Summer Topology Conference at CUNY Staten Island (July 2014) and
      the AMS Fall Southeastern Sectional Meeting special session on Set
      Theoretic Topology (November 2014)
  \item
    \emph{Finite and infinite games / Undergraduate research and grad school},
    invited presentation at Lamar University (June 2014)
  % \item
  %   \emph{Mathematics is all fun and games},
  %   presentations for
  %     Auburn University COSAM Graduate Student Colloquium (October 2013) and
  %     Auburn University DMS Graduate Student Colloquium (October 2013)
  % \item
  %   \emph{Limited information strategies for topological games},
  %   presentation for Auburn University Research Week (February 2013)
  \item
    Assorted presentations on set theory, game theory, and topology,
    for Auburn University REU in Algebra and Discrete Mathematics
      (Summers 2010-2015)
  \item
    \emph{The edge unfolding of generalized pyramids},
    presentation for the National Conference for Undergraduate Research
      (Spring 2008)
\end{itemize}


\vfill


\headerText{Coursework and Seminars}
\begin{multicols}{2}
\begin{itemize}
  \item
    General Topology
  \item
    Set Theoretic Topology
  \item
    Continuum Theory
  \item
    Axiomatic Set Theory
  \item
    Descriptive Set Theory
  \item
    Euclidean Geometry
  \item
    Game Theory
  \item
    Knot Theory
  \vfill
  \columnbreak
  \item
    Category Theory
  \item
    Graph Theory
  \item
    Enumeration
  \item
    Functions of Complex Variables
  \item
    Vietoris Homology
  \item
    Simplicial Homology
  \item
    Technology in Secondary\\ Mathematics Education
\end{itemize}
\end{multicols}


\newpage


\headerText{Teaching}
\begin{itemize}
  \item
    \textbf{Intermediate Algebra | MTH-098 (Southern Union)}

    Hybrid lecture/lab course in
    developmental algebra using the ALEKS learning management system.

  \item
    \textbf{Calculus for Engineering Technology | MATH-1121 (UNC Charlotte)}

    Topics covered: elements of differential and integral calculus for
    polynomial, rational, exponential, logarithmic and trigonometric
    functions, with applications to engineering.

  \item
    \textbf{(Honors) Calculus I | MATH-1610/1617 (Auburn University)}

    Topics covered: limits; the derivative of algebraic, trigonometric,
    exponential, and logarithmic functions; applications of the derivative,
    antiderivatives, the definite integral; applications to area problems;
    the fundamental theorem of calculus.

    For honors sections, students
    created a capstone project and presentation illustrating the application
    of calculus to their own field of study or interests.

  \item
    \textbf{Calculus II | MATH-1620 (Auburn University)}

    Topics covered: techniques of integration, applications of the integral,
    parametric equations, polar coordinates, vectors, lines and planes in space,
    infinite sequences, and series.

    Developed inquiry-based learning notes available on GitHub.

  \item
    \textbf{(Honors) Calculus III | MATH-2630/2637 (Auburn University)}

    Topics covered: vector-valued functions, partial derivatives,
    multiple integration, and vector calculus.

    For honors sections, students were assigned to research and present
    topics and examples during lecture (with optional assistance from the
    instructor).
    Developed inquiry-based learning notes available on GitHub.

  \item
    \textbf{Calculus IV | MATH-2242 (UNC Charlotte)}

    Topics covered: functions \(\mathbb R^n\to\mathbb R^m\), vector fields,
    line and surface integrals; Green's theorem, Divergence theorem,
    Stokes' theorem and applications.

  \item
    \textbf{Intermediate Euclidean Geometry I | MATH 5380 (Auburn University)}

    Topics covered: fundamental concepts and theorems of Euclidean geometry,
    introduction to higher dimensions; regular polygons and polyhedra,
    symmetry groups, convexity, geometric extremum problems; geometric
    transformations and their invariants.

    Used inquiry-based learning notes written by Andras Bezdek and
    Wlodzimierz Kuperberg.
\end{itemize}

\noindent
\textbf{Selected student comments (Auburn University 2014-2015):}
    \begin{itemize}
      \item ``Calc 2 is definitely a hard course but Mr. Clontz taught it in a way that made it easier to understand. Very glad he was my professor for the semester.''
      \item ``Fantastic teacher that had a very cool and unique teaching style that made everything very clear.''
      \item ``This was an interesting style [active learning] of class. I think that if it is implemented more frequently, it could really help future students understand and retain the material.''
      \item ``Great grading system. Great guy. Great class.''
    \end{itemize}



\vfill


\headerText{Mentorship/Advising}
\begin{itemize}
  \item
    \textbf{Tanyce James - Senior Project (UNC Charlotte, 2015-present)}

    An investigation of finite and infinite combinatorial games.
  \item
    Co-adviser of Pi Mu Epsilon UNCC chapter
\end{itemize}


\vfill
\newpage


\headerText{Outreach}
\begin{itemize}

  \item
    \textbf{MaPP (Mathematical Puzzle Programs)} (2015-present)

    Founded to bring mathematical puzzle competitions like LaMP and A.M.P.'d
    to campuses across the nation, and provide free mathematical puzzle
    materials for teachers to use in the classroom.
    Responsible for the MaPP High School
    Challenge and MaPP Middle School Challenge competitions to be ran
    throughout the 2016-2017 school year.
    Website at
    \href{http://www.mappmath.org}{\nolinkurl{MaPPmath.org}}.

  \item
    \textbf{LaMP (Lamar University Mathematical Puzzlehunt)} (2014-2015)

    Developed original mathematical puzzle-solving
    competition for high school students. Includes a logic-based physical
    challenge and several smaller mathematics puzzles, each of which gives
    clues for an overarching mathematical meta-puzzle.

  \item
    \textbf{AU Explore - Math EXPO} (2009-2013)

    Developed several twenty-minute workshop activities for fifth grade students
    involving number theory, game theory, geometry, and graph theory.
    Organized a volunteer staff of over a dozen graduate and undergraduate
    student volunteers to present these activities to rotating groups
    of students throughout the each annual event.

  \item
    \textbf{A.M.P.'d (Auburn Mathematical Puzzle) Challenge} (2012-2013)

    Co-created annual puzzlehunt-inspired mathematics competition for middle
    school students, serving as event coordinator and puzzle designer for the
    January 2012, September 2012, and September 2013 competitions.

  \item
    \textbf{War Eagle BEST Robotics Competition} (Judge, 2013)

  \item
    \textbf{AU Science Olympiad for Elementary School} (Event Designer, 2013)

  \item
    \textbf{AU Science Olympiad for Middle School} (Event Organizer, 2011-2012)
\end{itemize}


\vfill


\headerText{Cyberinfrastructure and Software Development}
\begin{itemize}
  \item
    \textbf{Online Seminars in Mathematics (OSM)}

    Initiative to bring online streaming of mathematical content
    using free and open-source software and minimal hardware.
    Project hosted at
    \href{http://osm.clontz.org}{\nolinkurl{osm.clontz.org}}.
  \item
    \textbf{$\pi$-Base Topology Database}

    Contributor and consultant. Developing undergraduate research
    experience to populate database with modern topological results.
    Collaborating on REU proposal based on the project
    with Ziqin Feng (Auburn University).
    Project hosted at
    \href{http://topology.jdabbs.com}{\nolinkurl{topology.jdabbs.com}}.
  \item
    \textbf{Teloga.com}

    Co-developed the customer relationship management (CRM) web application
    \href{http://teloga.com}{\nolinkurl{Teloga.com}} for collegiate
    music organizations.
  % \item
  %   \textbf{Global Urban Datafest Hackathon}

  %   Led team of four mathematicians in developing web application
  %   which analyzed footage of Toomer's Corner to programmatically
  %   determine if unusual activity is occurring.
  %   See the
  %   \href{https://istoomerscornerbeingrolledrightnow.github.io}{
  %   Toomer's Corner GitHub project page} for more details. Project
  %   received top honors in local competition.
  % \item
  %   \textbf{ALMS: Active Learning Management System}

  %   Developer for an open-source LMS based on Ruby on Rails and
  %   AngularJS for managing an active learning
  %   mathematics classroom, to be released in April 2015.
  \item
    \textbf{Open Source Software}

    Contributor to several OSS repositories in addition to the above
    through the active GitHub account
    \href{https://github.com/StevenClontz}{\nolinkurl{@StevenClontz}}.
  \item
    \textbf{Programming and Markup Languages}

    HTML5 (Markdown, HAML, Slim), CSS (LESS, SASS),
    Javascript (Coffeescript, jQuery, AngularJS),
    PHP (Wordpress, Wolf~CMS), Python (Django), Ruby (Rails, Sinatra, nanoc),
    \LaTeX{}, Git, Firebase, SQL (MySQL, PostgreSQL, SQLite),
    Data Serialization (JSON, YAML)
\end{itemize}


\vfill
% \newpage


% \headerText{Leadership and Service}
% \begin{itemize}
%   \item
%     \textbf{National Youth Leadership Training},
%       Boy Scouts of America (2002-2011)
%   \item
%     \textbf{Eagle Scout},
%       Boy Scouts of America (2004-present)
%   \item
%     \textbf{Freshman Adviser},
%       Auburn University Bands (2007-2008)
%   \item
%     \textbf{Vice President, Founding Member},
%       Auburn University Math Club (2008-2009)
%   \item
%     \textbf{President, Executive Board Member},
%       AU Graduate Student Council (2010-2012)
%   \item
%     \textbf{Graduate Student Peer Mentor},
%       AU Department of Mathematics and Statistics (2011-2013,2015)
%   \item
%     \textbf{Founding Member},
%       AU Mathematics and Statistics Graduate Student Leadership (2013)
% \end{itemize}

% \vfill


% \headerText{Other Experience}
% \begin{itemize}
%   \item
%     \textbf{Course Notes}

%     Contributor to inquiry-based learning algebraic topology notes written
%     by Krystina Kuperberg.
%     Author of course notes for Calculus II and Calculus III.
%   \item
%     \textbf{Learning Management Systems}

%     ALEKS, Instructure Canvas, Blackboard
%     %, and creator/developer of ALMS
%   \item
%     \textbf{Puzzle Design and Competition}

%     Has organized and competed in over a dozen separate puzzle competitions in
%     the Auburn area and abroad, and holds the longest winning streak in
%     Auburn puzzlehunt history.
%   \item
%     \textbf{Entrepreneurship}

%     Experience as a small business owner of Teloga, LLC, as well as a
%     freelance puzzle designer and mathematics tutor.
% \end{itemize}


% \vfill


\end{document}