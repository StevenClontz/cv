\documentclass[11pt]{amsart}
\usepackage{fancyhdr}
\usepackage[margin=1in]{geometry}
 
\usepackage[page]{totalcount}

\usepackage{amsthm}
\theoremstyle{plain}
\newtheorem{theorem}{Theorem}
\newtheorem{question}{Question}

\usepackage[pdfpagelabels]{hyperref}
\hypersetup{colorlinks=true,urlcolor=blue,linkcolor=black,citecolor=black}

% \usepackage{fontspec}
%   \defaultfontfeatures{Ligatures=TeX}
%   \setmainfont{Times New Roman}
% \usepackage{setspace}
% \doublespacing

\pagestyle{fancy} \headheight 15pt \footskip 20pt

\parskip=12pt

\rhead{\thepage/\totalpages}
\chead{}
\lhead{Clontz Research Statement}
\rfoot{\today}
\cfoot{}
\lfoot{}

\newcommand{\HRule}{\rule{\linewidth}{0.5mm}}


%%%%%%%%%%%%%%%
% Definitions %
%%%%%%%%%%%%%%%

% Strategy uparrow shortcuts
\newcommand{\win}{\uparrow}
\newcommand{\markwin}{\underset{\text{mark}}{\uparrow}}
\newcommand{\tactwin}{\underset{\text{tact}}{\uparrow}}
\newcommand{\kmarkwin}[1]{\underset{#1\text{-mark}}{\uparrow}}
\newcommand{\ktactwin}[1]{\underset{#1\text{-tact}}{\uparrow}}
\newcommand{\codewin}{\underset{\text{code}}{\uparrow}}
\newcommand{\limitwin}{\underset{\text{limit}}{\uparrow}}
\newcommand{\notwin}{\not\uparrow}
\newcommand{\notmarkwin}{\underset{\text{mark}}{\not\uparrow}}
\newcommand{\nottactwin}{\underset{\text{tact}}{\not\uparrow}}
\newcommand{\notkmarkwin}[1]{\underset{#1\text{-mark}}{\not\uparrow}}
\newcommand{\notktactwin}[1]{\underset{#1\text{-tact}}{\not\uparrow}}
\newcommand{\notcodewin}{\underset{\text{code}}{\not\uparrow}}
\newcommand{\notlimitwin}{\underset{\text{limit}}{\not\uparrow}}

\newcommand{\oneptcomp}[1]{#1^\star}
\newcommand{\oneptlind}[1]{#1^\dagger}
% \newcommand{\sharp}[1]{#1^{\#}}

% Games
\newcommand{\gruConGame}[2]{Gru_{O,P}(#1,#2)}

\newcommand{\gruKPGame}[1]{Gru_{K,P}(#1)}
\newcommand{\gruKLGame}[1]{Gru_{K,L}(#1)}

\newcommand{\cloPFGame}[1]{PtFin_{F,C}(#1)}

\newcommand{\menGame}[1]{Men_{C,F}(#1)}
\newcommand{\rothGame}[1]{Roth_{C,S}(#1)}
\newcommand{\rothAltGame}[1]{Roth_{P,O}(#1)}

\newcommand{\bellConGame}[1]{Bell_{D,P}(#1)}



\newcommand{\SigmaProdR}[1]{\Sigma\mathbb{R}^{#1}}
\newcommand{\sigmaprodtwo}[1]{\Sigma2^{#1}}

\newcommand{\concat}{{^\frown}}
\newcommand{\rest}{\restriction}

\newcommand{\cl}[1]{\overline{#1}}

\newcommand{\pow}[1]{\mc{P}(#1)}

\newcommand{\<}{\langle}
\renewcommand{\>}{\rangle}

\newcommand{\al}[1]{{#1}^*}

\newcommand{\mc}[1]{\mathcal{#1}}
\newcommand{\mb}[1]{\mathbb{#1}}

\newcommand{\po}{\mathbb{P}}
\newcommand{\pok}{\po_\kappa}

\newcommand{\Lim}{\mathrm{Lim}}
\newcommand{\Suc}{\mathrm{Suc}}

\newcommand{\ds}{\displaystyle}

\newcommand{\st}[2]{st\left(#1,#2\right)}

\newcommand{\alcomp}{\al\parallel}

\newcommand{\rank}{\textrm{rank}}
\newcommand{\dom}{\textrm{dom}}

\renewcommand{\mod}{\,\textrm{mod}}

\newcommand{\zip}{\bowtie}
\newcommand{\ran}[1]{\text{range}(#1)}

\newcommand{\cf}[1]{\textrm{cf}(#1)}

\newcommand{\alcompS}[1]{S(#1)}


\newcommand{\scish}{almost-$\sigma$-(relatively compact)}

\usepackage{mathrsfs}
\newcommand{\pl}[1]{\mathscr{#1}}



\newcommand{\term}{\textit}


\newcommand{\bakerGame}[1]{{Bak}_{A,B}(#1)}
\newcommand{\bmGame}[1]{{BM}_{E,N}(#1)}



%%%%%%%%%%%%
% Document %
%%%%%%%%%%%%


\begin{document}

\begin{center}

\textsc{\huge Steven Clontz}

% Title
\HRule \\[0.1cm]
{ \huge \bfseries Research Statement \\[0.4cm] }

\HRule \\[1.5cm]

\end{center}


My scholarship is primarily focused on a strong research program studying general topology,
particularly selection games. 
For context, a selection game is a two-player game that
characterizes the structure of a topological space based upon which player has a
winning strategy, a rule that a player can use to guarantee victory in the game
no matter how their opponent plays.
Key results published during my time at South
include the following. 

Theorem: A family of almost-compatible finite-to-one functions
mapping countable subsets of a cardinal
to the natural numbers cannot be used to construct a family of one-to-one
functions with the same property. [0]

This result has important ramifications in set theory and topology. In particular,
such a family of finite-to-one functions for the second uncountable cardinal
may be constructed using the usual ZFC axioms of 
set theory, but if the Continuum Hypothesis holds,
then it is impossible to construct such one-to-one functions. Considering the Menger
selection game, finite-to-one functions
are sufficient to design so-called ``Markov'' strategies that are able to track the number of moves
made in Menger's selection game, but one-to-one functions seem to be required to design
``tactical'' strategies that do not need to track this additional information.

Theorem: A Markov strategy for the second player in Menger's selection game 
may always be improved
to only require memory of the two most recent moves of the opponent. [1] 

This result is striking as it does not hold for another well-studied game
attributed to Banach and Mazur, characterizing Baire's topological property.
Telgarsky conjectures that there exists a space for which a winning
strategy using three moves of the opponent exists for the second player
in the Banach-Mazur game, but it cannot be
improved to a strategy using two moves of the opponent. My result demonstrates
why this conjecture is not a trivial assumption to make. 

Theorem: Any strategy for the second player in any selection game satisfying a
certain countability requirement may be improved to a Markov strategy requiring
memory of only the most recent move of the opponent. [2] 

This result improves the previous theorem whenever the countability requirement
is satisfied, as is the case in many applications. The importance of this
paper is witnessed by its inclusion in an upcoming special issue of Topology
and its Applications dedicated to selection principles and their applications.

In addition to the three papers cited above, I have two co-authored papers
currently under review and another preprint that I anticipate submitting later
this academic year, 
making my research productivity roughly two peer-reviewed papers a year. 
To date every mathematical research paper I have submitted for publication has been accepted
pending minor revisions, and the journals I publish in are the standard for publications
in my field. For comparison, my PhD advisor (Gary Gruenhage, Auburn University)
has published fifteen out of his eighteen research papers over the past decade
in these same journals. 

Beyond my publications, I am an active member in my scholarly communities.
I regularly attend either the Spring Topology and Dynamics Conference or
the Summer Conference on Topology and Its Applications each year to present
my work. I am a frequent referee for the journal Topology and Its Applications,
and reviewer for Mathematical Reviews.

While funding for mathematical research in my field is rare, I have actively
pursued both internal and external funding based upon my work in research,
teaching, and outreach. For research, my work as mathematical editor
for the pi-Base Database of Topological Spaces has particular promise for
external funding as a novel and cutting-edge development in my field.
I am regularly involved in proposals for NSF IUSE and other
external grants to support innovations
in undergraduate STEM education. Meanwhile, my work in mathematics outreach
via Mathematical Puzzle Programs has been funded by the Mathematical
Association of America, and I am regularly contacted and occassionaly consult
for organizations seeking my expertise in mathematical 
puzzle design. (My personal website
is the top result when searching for "types of puzzles" on Google.)

[0]: Clontz, Steven; Dow, Alan;
Almost compatible functions and infinite length games.
Rocky Mountain J. Math. 48 (2018), no. 2, 463–483.   

[1]: Clontz, Steven;
Applications of limited information strategies in Menger's game. 
Comment. Math. Univ. Carolin. 58 (2017), no. 2, 225–239. 

[2]: Clontz, Steven;
Relating games of Menger, countable fan tightness, and selective separability.
Topology Appl., to appear.

%\bibliographystyle{plain}
%\bibliography{bibliography}

\end{document}
