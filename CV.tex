\documentclass{article}
\usepackage[margin=1in]{geometry}
\usepackage{multicol}
\usepackage{float}
\usepackage{fancyhdr}
\usepackage{enumitem}

\usepackage[page]{totalcount}

\usepackage[colorlinks=true,urlcolor=blue]{hyperref}

\pagestyle{fancy} %\headheight 14.49998pt

\rhead{Clontz \thepage/\totalpages}
\chead{}
\lhead{Curriculum Vitae}
\rfoot{\today}
\cfoot{}
\lfoot{}

\newcommand{\HRule}{\rule{\linewidth}{0.5mm}}

\newcommand{\headerText}[1]{
  \noindent{\large \bf #1} \\*[-.8pc]
  \rule{\textwidth}{.1pt}}

\begin{document}

\thispagestyle{empty}





\begin{center}
{\Large Steven Craig Clontz, Jr.} \\[.5pc]
Auburn University Department of Mathematics \& Statistics \\
(256) 508-3864 $\;$
  \href{mailto:steven.clontz@gmail.com}{\nolinkurl{steven.clontz@gmail.com}} \\
\url{http://stevenclontz.com}
\end{center}


\vfill


\headerText{Education} \\
{\bf PhD in Mathematics, Auburn University}\hfill {\it May 2015} \\
Dissertation:
  \textit{Limited information strategies for topological games},
  under Gary Gruenhage\\
GPA: 4.00 \\
Fitzpatrick Fellow (2012-13, 2014-15), DMS Teaching Citation (2014-15) \\
COSAM Dean's Doctoral Research Award Nominee (2015) \\
\\
{\bf Masters in Mathematics, Auburn University}\hfill {\it December 2010} \\
Thesis:
  \textit{Applications of stationary sets in set theoretic topology},
  under Gary Gruenhage \\
GPA: 4.00 \\
\\
{\bf Bachelor of Science in Mathematics,
  Auburn University}\hfill {\it May 2008} \\
Honors Thesis:
  \textit{The edge unfolding of generalized pyramids},
  under Andras Bezdek \\
GPA: 3.88\\
Summa Cum Laude, University Honors Scholar, Dean's Medalist,
  Undergraduate Research Fellow \\
Phi Kappa Phi, Phi Beta Kappa\\


\vfill


\headerText{Specialties and Interests}
\begin{multicols}{2}
\begin{itemize}
  \item Set-theoretic topology
  \item Game theory
  \item Set theory
  \vfill
  \columnbreak
  \item Applications of software engineering \\
        to mathematics research and education
  \item Mathematical puzzles and games \\
        with applications to education and outreach
  \item Active and inquiry-based learning
\end{itemize}
\end{multicols}


\vfill


\headerText{Employment and Professional Experience}
\begin{itemize}
  \item
    \textbf{Graduate Teaching Assistant and Instructor},
      Auburn University Department of Mathematics
      (August 2008 - December 2013, August 2014 - May 2015)
  \item
    \textbf{Freelance mathematical puzzle and game designer}
      (June 2014 - present)
  \item
    \textbf{Specialist IV - Information Technology},
      Auburn University Office of University Writing
      (January 2014 - May 2014)
  \item
    \textbf{Mathematics Instructor},
      Southern Union State Community College
      (August 2013 - May 2014)
  \item
    \textbf{Co-founder and Software Engineer},
      Teloga LLC
      (November 2011 - July 2015)
  \item
    \textbf{Webmaster and Technology Assistant},
      Auburn University Bands
      (January 2008 - May 2010)
\end{itemize}


\vfill
\newpage


\headerText{Papers}
\begin{itemize}
  \item
    \textit{Proximal compact spaces are Corson compact.}
    Topology Appl. 173 (2014), 1–8 (with G. Gruenhage).
  \item
    \textit{On $k$-tactics for Gruenhage's compact-point game.}
    Submitted to Questions and Answers in Gemeral Topology
  \item
    \textit{Limited information strategies for Bell's proximal game.}
    In preparation.
  \item
    \textit{Game-theoretic strengthenings of Menger's property.}
    In preparation.
\end{itemize}


\vfill


\headerText{Presentations}
\begin{itemize}
  \item
    \emph{The edge unfolding of generalized pyramids},
    presentation for the National Conference for Undergraduate Research
      (Spring 2008)
  \item
    Assorted presentations on set theory, game theory, and topology,
    for Auburn University REU in Algebra and Discrete Mathematics
      (Summers 2010-2014)
  \item
    \emph{Limited information strategies for topological games},
    presentation for Auburn University Research Week (February 2013)
  \item
    \emph{Mathematics is all fun and games},
    presentations for
      Auburn University COSAM Graduate Student Colloquium (October 2013) and
      Auburn University DMS Graduate Student Colloquium (October 2013)
  \item
    \emph{Finite and infinite games / Undergraduate research and grad school},
    invited presentation at Lamar University (June 2014)
  \item
    \emph{Game-theoretic strengthenings of Menger's property},
    presentations for
      the 29th Summer Topology Conference at CUNY Staten Island (July 2014) and
      the AMS Fall Southeastern Sectional Meeting special session on Set
      Theoretic Topology (November 2014)
  \item
    \emph{Using AngularJS with Ruby on Rails},
    invited lecture for The Iron Yard (Atlanta) Ruby on Rails course
      (September 2014)
  \item
    \emph{Proximal compact spaces are Corson compact},
    presentation for the AMS/MAA Joint Mathematics Meetings at San Antonio, TX
      (January 2015)
  \item
    \emph{Limited information strategies for a topological proximal game},
    presentation for the AMS Sectional Mathematics Meeting at the University of
    Alabama at Huntsville
      (March 2015)
  \item
    \emph{Fun with Menger's game},
      Auburn University DMS Graduate Student Colloquium
        (April 2015)
\end{itemize}


\vfill


\headerText{Graduate Coursework and Seminars}
\begin{multicols}{2}
\begin{itemize}
  \item
    Intermediate Euclidean Geometry I
  \item
    Enumeration
  \item
    Technology in Secondary\\ Mathematics Education
  \item
    Topology I and II
  \item
    Game Theory
  \item
    Advanced Topics In Graph Theory\\ (two sections)
  \item
    Axiomatic Set Theory I and II
  \vfill
  \columnbreak
  \item
    Functions of Complex Variables I and II
  \item
    Real Functions And Set Theory I and II
  \item
    Set Theoretic Topology I and II
  \item
    Vietoris Homology
  \item
    Simplicial Homology
  \item
    Category Theory
  \item
    Set Theoretic Topology Seminar
  \item
    Continuum Theory Seminar
  \item
    Inverse Limits Seminar
\end{itemize}
\end{multicols}


\newpage


\headerText{Teaching}
\begin{itemize}
  \item
    \textbf{Intermediate Algebra - MTH-098 (Southern Union)}

    Hybrid lecture/lab course in
    developmental algebra using the ALEKS learning management system.

  \item
    \textbf{MathEXCEL (Auburn University)}

    Worksheet-based recitation course for students taking Calculus I.

  \item
    \textbf{(Honors) Calculus I - MATH-1610/1617 (Auburn University)}

    Topics covered: limits; the derivative of algebraic, trigonometric,
    exponential, and logarithmic functions; applications of the derivative,
    antiderivatives, the definite integral; applications to area problems;
    the fundamental theorem of calculus.

    For honors sections, students
    created a capstone project and presentation illustrating the application
    of calculus to their own field of study or interests.

  \item
    \textbf{Calculus II - MATH-1620 (Auburn University)}

    Topics covered: techniques of integration, applications of the integral,
    parametric equations, polar coordinates, vectors, lines and planes in space,
    infinite sequences, and series.

    Currently developing inquiry-based learning notes for Fall 2014.

  \item
    \textbf{(Honors) Calculus III - MATH-2630/2637 (Auburn University)}

    Topics covered: vector-valued functions, partial derivatives,
    multiple integration, and vector calculus.

    For honors sections, students were assigned to research and present
    topics and examples during lecture (with optional assistance from the
    instructor).

  \item
    \textbf{Intermediate Euclidean Geometry I - MATH 5380 (Auburn University)}

    Topics covered: Fundamental concepts and theorems of Euclidean geometry,
    introduction to higher dimensions. Regular polygons and polyhedra,
    symmetry groups, convexity, geometric extremum problems. Geometric
    transformations and their invariants.

    Used inquiry-based learning notes written by Andras Bezdek and
    Wlodzimierz Kuperberg.

  \item
    Various tutoring experience as assigned by the AU Mathematics Department
    for calculus and analysis, in addition to freelance work as a
    college-preparatory and university-level mathematics tutor.

\end{itemize}


\vfill


\headerText{Outreach}
\begin{itemize}
  \item
    \textbf{A.M.P.'d (Auburn Mathematical Puzzle) Challenge} (2012-2013)

    Co-created annual puzzlehunt-inspired mathematics competition for seventh
    and eighth
    grade students, serving as event Coordinator and puzzle designer for the
    January 2012, September 2012, and September 2013 competitions.

    Served as writer, director, actor, videographer, and editor for videos
    framing the scenario for the competition, as well as designing \LaTeX{}/PDF
    documentation to match the theme.

    Wrote several mathematical puzzles based on graph theory,
    design theory, game theory, geometry, and other fields to be solved by
    teams of six to eight students.

    Coordinated a staff of 35 graduate and undergraduate student volunteers
    and AU COSAM Outreach leadership each year.

  \item
    \textbf{AU Explore - Math EXPO} (2009-2013)

    Developed several twenty-minute workshop activities for fifth grade students
    involving number theory, game theory, geometry, and graph theory.

    Organized a volunteer staff of over a dozen graduate and undergraduate
    student volunteers to present these activities to rotating groups
    of students throughout the each annual event.

  \item
    \textbf{Lamar University Mathematical Puzzlehunt} (2014-2015)

    Served as paid consultant for developing mathematical puzzle-solving
    competition for high school students. Includes a logic-based physical
    challenge and several smaller mathematics puzzles, each of which gives
    clues for an overarching mathematical meta-puzzle.

  \item
    \textbf{War Eagle BEST Robotics Competition} (Judge, 2013)

  \item
    \textbf{AU Science Olympiad for Elementary School} (Event Designer, 2013)

  \item
    \textbf{AU Science Olympiad for Middle School} (Event Organizer, 2011-2012)
\end{itemize}


\vfill


\headerText{Software Development}
\begin{itemize}
  \item
    \textbf{Programming and Markup Languages}

    HTML5 (Markdown, HAML, Slim), CSS (LESS, SASS),
    Javascript (Coffeescript, jQuery, AngularJS),
    PHP (Wordpress, Wolf~CMS), Python (Django), Ruby (Rails, Sinatra, nanoc),
    \LaTeX{}, Git, Firebase, SQL (MySQL, PostgreSQL, SQLite),
    Data Serialization (JSON, YAML)
  \item
    \textbf{Teloga.com}

    Co-founded Teloga, LLC to manage the customer relationship management
    website
    \href{https://teloga.com}{\nolinkurl{Teloga.com}} for music organizations,
    based on the Ruby on Rails and AngularJS frameworks (formerly Django).
  \item
    \textbf{Global Urban Datafest Hackathon}

    Led team of four mathematicians in developing web application
    which analyzes footage of Toomer's Corner to programmatically
    determine if unusual activity is occurring.
    See the
    \href{https://istoomerscornerbeingrolledrightnow.github.io}{
    Toomer's Corner GitHub project page} for more details. Project
    received first place at the local competition and is in consideration
    for global competition awards.
  \item
    \textbf{$\pi$-Base Topology Database}

    Database contributor and front-end code consultant. Project hosted at
    \url{http://topology.jdabbs.com}.
  \item
    \textbf{ALMS: Active Learning Management System}

    Developer for an open-source LMS based on Ruby on Rails and
    AngularJS for managing an active learning
    mathematics classroom, to be released in April 2015.
  \item
    \textbf{Open Source Software}

    Contributor to several OSS repositories in addition to the above
    through the active GitHub account
    \href{https://github.com/StevenClontz}{\nolinkurl{@StevenClontz}}.
\end{itemize}


\vfill


\headerText{Leadership and Service}
\begin{itemize}
  \item
    \textbf{National Youth Leadership Training},
      Boy Scouts of America (2002-2011)
  \item
    \textbf{Eagle Scout},
      Boy Scouts of America (2004-present)
  \item
    \textbf{Freshman Adviser},
      Auburn University Bands (2007-2008)
  \item
    \textbf{Vice President, Founding Member},
      Auburn University Math Club (2008-2009)
  \item
    \textbf{President, Executive Board Member},
      AU Graduate Student Council (2010-2012)
  \item
    \textbf{Graduate Student Peer Mentor},
      AU Department of Mathematics and Statistics (2011-2013,2015)
  \item
    \textbf{Founding Member},
      AU Mathematics and Statistics Graduate Student Leadership (2013)
\end{itemize}

\vfill
\newpage


\headerText{Other Experience}
\begin{itemize}
  \item
    \textbf{Course Notes}

    Contributor to inquiry-based learning algebraic topology notes written
    by Krystina Kuperberg.
    Author of course notes for Calculus II and Calculus III.
  \item
    \textbf{Learning Management Systems}

    ALEKS, Instructure Canvas, Blackboard, and creator/developer of ALMS
  \item
    \textbf{Puzzle Design and Competition}

    Has organized and competed in over a dozen separate puzzle competitions in
    the Auburn area and abroad, and holds the longest winning streak in
    Auburn puzzlehunt history.
  \item
    \textbf{Entrepreneurship}

    Experience as a small business owner of Teloga, LLC, as well as a
    freelance puzzle designer and mathematics tutor.
\end{itemize}


\vfill


\end{document}